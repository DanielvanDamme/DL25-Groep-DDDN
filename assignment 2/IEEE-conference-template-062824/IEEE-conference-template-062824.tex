\documentclass[conference]{IEEEtran}
\IEEEoverridecommandlockouts

\usepackage{cite}
\usepackage{hyperref}
\usepackage{url}
\usepackage{subcaption}
\usepackage{float}
\usepackage{placeins}
\usepackage{svg}
\usepackage{amsmath,amssymb,amsfonts}
\usepackage{algorithmic}
\usepackage{graphicx}
\usepackage{textcomp}
\usepackage{xcolor}
\def\BibTeX{{\rm B\kern-.05em\textsc{i}\kern-.025em b\kern-.08em
    T\kern-.1667em\lower.7ex\hbox{E}\kern-.125emX}}
\begin{document}

\title{TODO}

\author{\IEEEauthorblockN{Dani\"el Jochems}
\IEEEauthorblockN{David Huizinga}
\IEEEauthorblockN{Niek Grimbergen}
\IEEEauthorblockN{Daan van Dam}}

\maketitle

\begin{abstract}
\end{abstract}

\section{Introduction}

\section{Methods}

\subsection{Data Engineering}

\subsubsection{Normalization}
To prepare the data for the model, we implemented a normalization procedure using the Normalizer 
class from sklearn. The normalize function applies L2 normalization to both the training and test 
datasets. With L2 normalization, each vector is scaled so that its norm equals 1, while preserving 
the direction of the original data. This helps reduce differences in scal or magnitude across the 
different sensor readings and labels, this will improve the models' learning process. The normalize
function works by first fitting separate normalizers on the training data for X and y. These fitted 
normalizers are then used to transform both the training and test sets so that the test data is 
placed on the same scale as the training data.

\subsubsection{Downsampling}

\subsection{Model Description}

\subsection{Hyperparameter Tuning}

\section{Results}

\section{Discussion}
An attempt was made to do some specific MEG preprocessing like namely, a fixed bandpass filter, prepare
channel z-scoring, and baseline correction. The inspiration for these came from the 
\href{https://mne.tools/stable/index.html}{MNE tools} package and the corresponding paper by Gramfort et al 
\cite{gramfort2013meg}. This applied preprocessing before model training severaly impacted results.
Together they destroyed the signal for effective classification, rendering the model useless at 
classifying. per channel z-scoring alone was effective.

\section*{AI Statement}

\bibliographystyle{IEEEtran}
\bibliography{references}

\section*{Supplementary Materials}
[Placeholder for supplementary figures or results]

\clearpage
\onecolumn

\end{document}