\documentclass[conference]{IEEEtran}
\IEEEoverridecommandlockouts
% The preceding line is only needed to identify funding in the first footnote. If that is unneeded, please comment it out.
%Template version as of 6/27/2024

\usepackage{cite}
\usepackage{amsmath,amssymb,amsfonts}
\usepackage{algorithmic}
\usepackage{graphicx}
\usepackage{textcomp}
\usepackage{xcolor}
\def\BibTeX{{\rm B\kern-.05em{\sc i\kern-.025em b}\kern-.08em
    T\kern-.1667em\lower.7ex\hbox{E}\kern-.125emX}}
\begin{document}

\title{TODO}

\author{\IEEEauthorblockN{Daniël Jochems}
\IEEEauthorblockA{\textit{TODO (student number)}}
\and
\IEEEauthorblockN{David}
\IEEEauthorblockA{TODO}
\and
\IEEEauthorblockN{Niek Grimbergen}
\IEEEauthorblockA{TODO}
\and
\IEEEauthorblockN{Daan van Dam}
\IEEEauthorblockA{6434657}
}

\maketitle

\begin{abstract}
TODO: Most important paragraph. Context (our question, the problem statement), content (here we go 
over what we did and what we found), conclusion (why/how it matters). Max 150 words.
\end{abstract}

\begin{IEEEkeywords}
TODO
\end{IEEEkeywords}

\section{Introduction}
TODO: Why does the paper matter? Set up gap in science/knowledge focusing down from a bigger 
problem to our smaller problem. Last paragraph should summarize our results to fill this gap. Here 
we can give context and background if we want than the abstract. We cannot give a conclusion, at 
most preview/tease the conclusion/results.

\section{Methods}
TODO: our approach and subsections for why we use that approach.

\section{Results}
TODO: This should be a sequence of statements each supported with some evidence and a conclusion.
Each statement is a subsection, the first paragraph of a statement summarizes the overall approach 
to solving the problem stated in the introduction, explain essential features of existing methods
we use and necessary background information. If we made a method ourselvers point it out clearly. 
Figures must be self contained (the legend and caption should have all the info for the figure).

\section{Discussion}
TODO: First we discuss how the gap from the introduction was filled by the report, the limitations
of the approach, some future directions and perhaps an open problem we find. Afterwards Basically 
everything we put some time into but did not write out to a result can be put here and anything we 
wanted to do but did not can also be put here.

\section*{AI statement}
TODO: If we used machine learning generative AI to do anything we put it here.

\section*{References}

\end{document}
